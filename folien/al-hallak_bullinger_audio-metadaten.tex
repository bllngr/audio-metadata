\documentclass[14pt]{beamer}

\usepackage[ngerman]{babel}
\usepackage[utf8x]{inputenc}
\usepackage{ucs}
\usepackage{BeamerColor}
\usepackage[babel,german=guillemets]{csquotes}
\usepackage[T1]{fontenc}
\usepackage{lmodern}
\PrerenderUnicode{Ü,ü}
\usepackage{amssymb,amsmath}

\usepackage{graphicx}
\usepackage{epstopdf}

\setbeamertemplate{footline}[frame number]
\usetheme{Boadilla}
\usecolortheme[named=chartreuse4]{structure}
\setbeamerfont{penis}{size*={9.0}{12.00}}
\setbeamerfont{quellen}{size*={6.0}{10.00}}
\renewcommand{\arraystretch}{2}

\beamertemplatenavigationsymbolsempty
\setbeamertemplate{itemize item}[triangle]
\setbeamertemplate{itemize subitem}[triangle]

\setbeamertemplate{sections/subsections in toc}[square]{}

\setbeamertemplate{blocks}[rounded][shadow=false]

\title{Metadaten für Audio-/Radiosignale}
\subtitle{Übersicht zu Standards und zur Gewinnung}

\author[M. Al-hallak, J. Bullinger]
{Malik Al-hallak \and Julius Bullinger}

\institute[Uni Weimar]{Bauhaus-Universität Weimar \\
Fakultät Medien \\
Bereich Vernetzte Medien \\
Audiotechnik}
\date{24. Juni 2011}

\begin{document}

\begin{frame}
	\titlepage
\end{frame}

\begin{frame}
	\frametitle{Motivation}
	
	Warum:
		\begin{itemize}
			\pause \item ermöglicht die Suche nach Audioinhalten
			\pause \item internationaler standardisierter Austausch von multimedialen Ressourcen
			\pause \item Information über Ressource ohne deren Inhalt zu betrachten\\
		\end{itemize}
		\pause 
	\vspace{.5cm}Vorraussetzung:
		\begin{itemize}
			\pause \item Standardisierung
			\pause \item automatisches oder manuelles Generieren
		\end{itemize}

\end{frame}

\begin{frame}
	\frametitle{Metadaten-Standards}
	\framesubtitle{Übersicht (Auswahl)}
\textbf{EBUCore MDS:}
	\begin{itemize}
		\pause \item European Broadcasting Union
		\pause \item Erweiterung des Dublin Core Metadata Set
		\pause \item Spezifiziert minimale Liste von Attributen, die Medieninhalte charakterisieren (nötige Information zum Suchen und Tauschen medialer Daten)
	\end{itemize}
\end{frame}



\begin{frame}
	\frametitle{Metadaten-Standards}
	\framesubtitle{Übersicht (Auswahl)}
	
	\textbf{MPEG-7 MDS:} \\
	\begin{itemize}
		\pause \item \emph{Moving Picture Expert Group}, 1996--2002
		\pause \item Analysewerkzeuge zur \textbf{technischen} Beschreibung der Ressourcen bis zu einem hohen Abstraktionsgrad
		\pause \item geeignet für interpretierende Suche
	\end{itemize}
\end{frame}

\begin{frame}
	\frametitle{Metadaten-Standards}
	\framesubtitle{Übersicht (Auswahl)}
	 \textbf{IPTC News Architecture G2:}
	\begin{itemize}
		\pause \item \emph{International Press Telecommunications Council}, 2000--2008
		\pause \item Repräsentation und Verwaltung von Nachrichten (NewsML), Events (EventsML) und Sportveranstaltungen (SportsML)
		\pause \item Paketfähigkeit: Mehrere NewsItems in einer Instanz; kein Inhaltsformat
	\end{itemize}
\end{frame}

\begin{frame}
	\frametitle{Metadaten-Standards}
	\framesubtitle{Übersicht (Auswahl)}
	\textbf{TV-Anytime:}
	\begin{itemize}
		\pause \item Vereinigung von mehr als 60 Organisationen in Europa, Asien und den USA
		\pause \item Homogenisierung vorhandener (proprietärer) Metadaten
		\pause \item Spezifikation zur Verbreitung und Verarbeitung von
		 Metadaten für den Endverbraucher ($\uparrow$PVR, $\uparrow$EPG)
	\end{itemize}
\end{frame}

\begin{frame}{Vergleich}
	\usebeamerfont{penis}
	\begin{tabular}{ p{2.3cm} || p{1.9cm} | p{2cm} | p{1.9cm} | p{1.8cm}}
	 & \centering{MPEG-7} & \centering{NewsML-G2} & \centering{EBUCore} & TV-Anytime \\ \hline  \hline

	\centering{Nutzer}  & \centering{Algorithmen} & \centering{Nachrichten-agenturen} & \centering{Europeana EUScreen} & Verbraucher \\ \hline
	\centering{Abstraktionsgrad} & \centering{sehr hoch} & \multicolumn{2}{c|}{gering} & \hspace{.6cm}hoch \\ \hline
	\centering{Generierung} & \centering{automatisch} & \multicolumn{3}{c}{manuell} \\ \hline
		\centering{Stelle in der Verwertungskette} & \centering{Suchmaschine (Verbraucher)} & \centering{Nach der Produktion} & \centering{Gesamte Ver-wertungskette} & Bei der Ausstrahlung  \\ \hline
	\centering{Zweck} & \centering{Intuitive \hspace{.5cm} Suche} & \centering{Austausch von Nachrichten} & \centering{Austausch, Publikation, Archivierung} & \centering{Informationen für den End-verbraucher} 
	\end{tabular}
\end{frame}


\begin{frame}
	\frametitle{MAWG}
	\framesubtitle{W3C Media Annotations Working Group (work in progress)}
	\begin{itemize}
		\pause \item \enquote{Meta-Standard} zur Zusammenführung genannter Standards (und weiteren)
		\pause \item Möglichkeiten, von einem Standard in einen anderen zu konvertieren
		\pause \item Gemeinsame Obermenge für alle Standards
		\pause \item APIs für Browser via JavaScript und RDF
		\pause \item \textbf{Keine} vollständige Spezifikation eines neuen Metadaten-Standards
	\end{itemize}
\end{frame}

\begin{frame}
	\frametitle{Quellen}
	\usebeamerfont{quellen}
	\begin{itemize}
		\usebeamerfont{quellen}
	\item Metadaten für Audiosignale:
	
	\begin{itemize}
	\usebeamerfont{quellen}
	\item http://www.dlib.indiana.edu/\textasciitilde jenlrile/metadatamap/seeingstandards\_glossary\_pamphlet.pdf
	\item  http://is-frankfurt.de/veranstaltung/Groffmann\_SS05/Thema\%205\_Broeder\_dl.pdf
	\end{itemize}
	
	\item TV-Anytime:
	
	\begin{itemize}
		\usebeamerfont{quellen}
	\item  http://tech.ebu.ch/Jahia/site/tech/cache/offonce/tvascope
	\item http://www.etsi.org/WebSite/Technologies/TVAnytime.aspx
	\end{itemize}
	
	\item EBUCore:
	
	\begin{itemize}
		\usebeamerfont{quellen}
	\item  http://tech.ebu.ch/docs/tech/tech3293v1\_2.pdf
	\item http://tech.ebu.ch/MetadataSpecifications
	\end{itemize}
	
	\item IPTC:
	
	\begin{itemize}
		\usebeamerfont{quellen}
		\item Laurent Le Meur: NewsML-G2 and the IPTC News Architecture G2
	\end{itemize} 
	\item MPEG-7:
		\begin{itemize}
			\usebeamerfont{quellen}
			\item http://www2.tu-ilmenau.de/mediaevent/archiv/fktg/regionalveranstaltungen/Skripte/mpeg7.PDF
			\item http://mpeg.chiariglione.org/standards/mpeg-7/mpeg-7.htm
			\item http://www.techfak.uni-bielefeld.de/~tkaempfe/lehre/CBIR/IDBausarbeitungen/MPEG-7.pdf

			
		\end{itemize}
		
	\end{itemize}


\end{frame}

\end{document} 